\begin{doublespace}

\chapter*{Introduction}
\addcontentsline{toc}{chapter}{Introduction}

\section*{Background}
\addcontentsline{toc}{section}{Background}

Richard Feynman once said, ``If our small minds, for some convenience,
divide this \ldots\ universe, into parts---%
physics, biology, geology, astronomy, psychology, and so on---%
remember that nature does not know it.''
%
`Biology' is nothing more than a convenient term used to categorize
certain kinds of complex natural patterns and processes.
%
Many of these processes, however, are not phenomena
that strictly occur in living systems.
%
For example, the process of diffusion in the electron transport chain,
which generates energy for cells in the form of ATP molecules,
is used by biological systems but is not exclusively a biological process.
%
Diffusion is a natural phenomenon that occurs whenever
particles exhibit Brownian, or random, motion,
and was, in fact, first studied by physical scientists.
%
In the same way, evolution via natural selection occurs in biological systems,
but it is not exclusively a biological process \citep{pen07}.
%
Cultural phenomena, such as languages and ideas,
and artificial genetic systems, which I discuss later,
can also evolve via selection.
%
Natural selection is a universal process that occurs whenever
three conditions are met: (1)~inheritance, (2)~variation,
and (3)~differential survival/reproduction \citep{ada06}.



Therefore, the process of evolution can be studied in isolation.
%
Mathematical models and computer simulations of evolutionary processes
show that it can be studied outside of the biological realm.
%
Using such models I have learned a great deal about evolution,
creating predictions that can then be tested in biological organisms.
%
However, models are limited in that they cannot capture
the complexity and open-endedness of evolution in action \citep{yed01}.
%
For example, models often deal with only a range of parameters values
and their possible outcomes is often limited in scope,
so that new patterns or new behavior is not possible \citep{yed01}.
%
Biological models, such as the use of \emph{E. coli} and \emph{S. cerevisiae},
have provided great insights about the working of evolution,
but even these systems can be intractable.
%
A middle-ground between these extremes---mathematical models
and biological ones---is the use of artificial life systems \citep{yed01}.
%
These are systems in which the conditions for evolution are present,
and therefore constitute an instance of evolution \citep{pen07}.



In my dissertation, I use the artificial life system Avida \citep{ofr04}
to study specific genetic mechanisms of adaptation and speciation.
%
Digital populations in Avida meet the conditions required for evolution;
therefore, Avida represents an instance of evolution.
%
I discuss how Avida works in more detail in the next section, but briefly,
digital organisms consist of a sequence of instructions (a `genome')
that is passed on to offspring during replication (meets inheritance condition).
%
Phenotypic and genotypic variation is introduced through mutations
and, in sexual populations, through recombination.
%
Organisms with phenotypes that match their environment
are able to reproduce faster than others
(meets differential survival/reproduction condition).
%
Avida is unlike a mathematical or computational model
in that it need not be constrained by parameters and their ranges;
evolving digital organisms evolve novelty and complexity
that is impossible to have been predicted \citep{yed01,wil02}.
%
(Note: \citet{yed01} investigated the Terra system,
from which Avida was based, but I will sometimes cite them
when discussing Avida if it is appropriate.)
%
Indeed, the patterns observed in Avida runs can be so complex
as to require traditional models in order to understand them
(B. \O stman, pers. comm.)
%
Avida thus provides an independent system in which predictions
made from evolutionary theory could be tested \citep{yed01},
and in turn provide feedback to those theories
and help refine them \citep{wil02}.



Apart from providing open-ended evolution and potentially novel patterns,
Avida has other advantages due to its computational nature.
%
In Avida, one has complete and accurate knowledge of individual organisms,
their genotype, fitness, and lineage, and one can track individual mutations.
%
This power allows one to create experiments of unprecedented sophistication
that would have been extremely difficult or even impossible
to carry out in biological organisms \citep{ele03}.
%
For example, one is able to go back to any point in time after an experiment,
change the value of a variable, and re-run the experiment
in exactly the same way except for the altered variable.
%
Another advantage is the capacity to run exact replicates
of an experimental configuration
(not counting the initial random `seed,' of course),
providing high statistical power to data analyses.
%
Using Avida, one is able to run experiments for millions of generations
in a matter of days or weeks.



These benefits are not without a cost.
%
The most prominent is that one is restricted to the kinds of questions
to which Avida is best suited \citep{wil02}.
%
For example, it would not make sense to study
the evolution of mitosis (Avida has no true cell division).
%
Thus, the questions asked should be amenable to abstraction \citep{wil02}.
%
Additionally, one may sometimes `miss the forest for the trees'
when reporting or interpreting Avida results,
particularly when the purpose is to study biological phenomena.
%
Special care must be taken to ensure that observations
are not merely artifacts of the computational system
in order to make the appropriate biological inferences.
%
Being a computational system, Avida requires a certain level
of computer proficiency, especially when performing complex data analyses
or when customizations to the system itself are needed.



Physiologically, digital organisms in Avida work very differently
from biological organisms \citep{yed01},
and they are not as complex as even the simplest lifeforms \citep{len01}.
%
However, several evolutionary properties have been found
to be remarkably similar to that of biological organisms
\citep{yed01,wil02,ada06}
(e.g., the distribution of mutational effects, the types of epistasis,
and the genetic architecture of sexual organisms).
%
But as R. Lenski points out in \citet{one03},
``even if the digital and biological realms sometimes come
into scientific conflict, it would only lead one to ask why
and then probe the relevant factors more deeply.''
%
Therefore, Avida provides an additional avenue of inquiry
where experiments can be conducted, and the results compared
to those of model organisms and theory,
in order to discover the generality of some phenomenon \citep{wil02}.



In addition to all the benefits that Avida offers to biologists,
Avida is also of interest to engineers.
%
Other evolutionary computational tools, such as genetic algorithms,
have been used by engineers to tackle difficult problems
that are best solved by evolution rather than by design \citep{mck08}.
%
As in applications to biological questions,
Avida offers a more open-ended approach to optimization and
algorithms for solving engineering problems \citep{mck08}.
%
Evolutionary concepts, such as robustness, evolvability, and cooperation,
are very interesting to system and software engineers,
who would like to develop systems that can compensate for failures,
be resilient to varying parameters, protect themselves from attacks,
gather information cooperatively,
and be efficient at distributing data through a network
\citep{bec07,mck08,gol08,kno09,kno11}.
%
\citet{wil02} posit that ``robots, and the software that directs them,
might evolve without human interaction, at which point they would become
part of the ecosystem in which I live.''
%
Using Avida, biologists themselves can therefore contribute to solving problems
outside their field, broadening their impact of their research \citep{one03}.



In this dissertation, I use Avida to investigate
some genetic mechanisms of adaptation and speciation.
%
Adaptation is the process in which beneficial traits
are acquired by organisms in a population through time.
%
Adaptive traits driven to fixation by natural selection
are ultimately encoded in the genetic material of organisms,
and therefore adaptation is often thought of in terms of adaptive alleles.
%
One of the goals of research in evolutionary biology
is to understand the genetic basis of adaptation.
%
For example, does the raw material for adaptive evolution
come from from new mutations or from ``standing genetic variation''
(i.e., allelic variation present in the population) \citep{orr05}?
%
Questions like this require detailed analyses of genetic data,
such as tracking individual alleles through time \citep{bar08}.



Speciation, the process by which new species form,
is often a by-product of adaptation \citep{coy04,sch09,sob10}.
%
Populations that evolve independently are likely to adapt in different ways
and therefore diverge genetically.
%
Genetic differences are likely to create ecogeographic, morphological,
physiological, or genetic isolation between populations \citep{coy04,sch10},
collectively known as ``reproductive isolation.''
%
For example, populations may not recognize each other as potential mates
or may produce sterile or inviable hybrids.
%
A topic of much recent consideration
is the genetic mechanisms that lead to reproductive isolation.
%
Like in studies of the genetic basis of adaptation,
an understanding of the genetic basis of speciation
requires detailed analyses of genetic data over time.



The topic of speciation, as commonly defined as reproductive isolation,
has never before been studied in Avida.
%
This was because sexual reproduction in Avida has only recently
been implemented, originally to study the evolution of sex
\citep{mis04,mis06,mis10}.
%
Sexual reproduction in Avida was implemented by exchanging genetic material
between the next two organisms that are ready to reproduce.
%
Currently, there is no mate recognition in Avida
(although research in this topic has been done),
and therefore prezygotic isolation is not easy to study.
%
My research on speciation has thus focused on postzygotic isolation,
i.e., hybrid sterility or inviability,
as creating hybrids between organisms and measuring their fitness is simple.
%
My research has demonstrated that Avida is a useful tool to complement
other approaches in speciation research as it allows for
the direct observation of evolution and reproductive isolation in action.



\section*{Study system: Avida}
\addcontentsline{toc}{section}{Study system: Avida}
\label{sec:avida}

In this section, I provide a brief overview of Avida;
for a full description, see \cite{ofr04}.
%
Avida is freely available at \url{http://avida.devosoft.org}.
%
In Avida, digital organisms are composed of
a linear sequence of instructions (akin to a haploid genome),
memory space in the form of registers and stacks,
pointers to memory locations,
and a central processing unit (CPU) that executes instructions.
%
The instruction set makes up an assembly-like programming language,
consisting of instructions for arithmetic operations,
memory manipulation (e.g., swapping registers or pushing into a stack),
conditional execution, iteration, input/output operations,
and allocation and copying of memory.
%
Organisms execute their instructions sequentially,
sometimes skipping instructions for conditional statements
or repeating the same instructions inside a loop;
when the last instruction is executed,
execution starts again at the first instruction.
%
By executing instructions in their genomes,
organisms are able to (1) replicate and (2) perform computational `tasks'
that increase the speed at which they replicate and thus increase fitness.



To replicate, an allocation instruction creates the memory space
required by the organism's offspring, and a copy instruction inside a loop
allows the organism to copy itself into the new memory space.
%
The copy instruction that allows organisms to replicate
has a configurable probability of making mistakes,
which introduces various kinds of mutations.
%
By default, replication is asexual.
%
However, Avida may be configured to perform sexual replication,
in which the genomes of two asexually-produced offspring are recombined
by exchanging two randomly-sized regions of their genomes.
%
The offspring (whether clonal or recombinants) are put into the population
in random locations, replacing whatever organisms were already there.
%
Generations are therefore overlapping, as offspring are born continuously,
replacing older individuals but who are likely not their parents.



In addition to replication, genomic instructions allow organisms
to acquire 32-bit input values and use them to perform computational tasks.
%
Tasks are boolean operations, such as NOT, AND, and OR,
and are applied to input values bit by bit.
%
For example, if input values were 8 bits,
the operation 10011101 AND 11101011 would produce 10001001
according to the rules of boolean logic for AND
(0~AND~0~=~0, 0~AND~1~=~0, 1~AND~0~=~0, and 1~AND~1~=~1.
%
In Avida, however, there is no AND operation nor any other boolean operation
except for NAND, from which all other boolean operations may be built,
a property of NAND known as `functional completeness' in boolean algebra.
%
For example, P~AND~Q, where P and Q are input values,
is equivalent to (P~NAND~Q)~NAND~(P~NAND~Q).
%
Therefore, in order to perform a task other than NAND,
digital organisms must make use of other instructions available,
which arise through mutation or recombination.



When an organism performs a task, the organism's `merit' is increased
by a specific amount, specified in a configuration file, for that task.
%
The merit of an organism is a unitless value used by Avida
to determine the number of instructions an organism may execute each time step.
%
If two organisms had the same merit,
they would execute the same number of instructions at each time step;
however, if one organism had twice the merit as another,
the first organism would execute twice the number of instructions
compared to the second in a single time step.
%
Thus, an organism with twice the merit as another
would replicate twice as fast.
%
Organisms initially inherit the merit of their parents;
otherwise, new organisms would be at a disadvantage
compared to the rest of the population.
%
The default environment rewards for nine binary (i.e., two-input) tasks.



Adaptation in Avida occurs naturally (i.e., it is not simulated),
as a result of the three ingredients required for natural selection:
inheritance, variation, and differential reproduction.
%
Inheritance comes from replication (sexual or asexual),
variation comes from mutation and recombination,
and differential reproduction comes from their rate of replication
(determined by their replication code and performance of tasks).
%
The ability to perform tasks evolves as organisms with the right mutations
replicate faster than others and therefore take over the population.
%
There are many ways in which to perform any one task,
and independently evolved organisms often evolve the same task
in different ways and with different degrees of efficiency.



\section*{Dissertation summary}
\addcontentsline{toc}{section}{Dissertation summary}

Evolutionary adaptation to a new environment
depends on the availability of beneficial alleles.
%
Beneficial alleles may appear as new mutations
or may come from standing genetic variation---%
alleles already present in the population
prior to the environmental change.
%
Adaptation from standing genetic variation
in sexually-reproducing populations
is expected to be faster than from new mutations
because beneficial alleles from standing genetic variation
occur at a higher starting frequency and are immediately available.
%
The distribution of fitness effects of alleles
from standing genetic variation are expected to be different
from that of new mutations because standing genetic variation
has been `pre-tested' by selection.
%
Whether adaptation uses standing genetic variation
or new mutations as a source of beneficial alleles is unknown.
%
In Chapter~\ref{chap:sgv} (p.~\pageref{chap:sgv}),
I conducted experimental evolution of digital organisms
to determine the source of beneficial alleles during adaptation.
%
I also tested the speed of adaptation
and the fitness effect of alleles
under these two sources of genetic variation.



Various processes may drive a biological population to acquire mutations
that reduce its fitness (i.e., ``deleterious mutations'').
%
A population with deleterious mutations may fully or partially recover
in fitness in two ways: reversion or compensatory adaptation.
%
Compensatory adaptation is the fixation of mutations
that ameliorate the effects of deleterious mutations
while the original deleterious mutations remain fixed.
%
Reversion is often the best way to recover and,
if a revertant mutation were to appear, the most probable route.
%
However, it has been found experimentally that
there are more compensatory mutations available
than the single revertant.
%
It has also been found that once compensatory adaptation
has begun, reversion becomes an increasingly improbable route
because the effect of the reversion changes with the genetic background.
%
Therefore, it seems that reversion has a limited
window of opportunity in which its full effect would be beneficial.
%
The fitness effect of the deleterious mutation,
the population size, and the mutation rate are three main factors
that will affect this window of opportunity.
%
In Chapter~\ref{chap:comp_rate} (p.~\pageref{chap:comp_rate}),
I used populations of digital organisms
to investigate the effect of these factors.
%
I found that the lower the initial fitness of the population,
the smaller the window of opportunity for reversion.
%
This result was partly caused by the stronger negative interactions
between compensatory mutations and the reversion.
%
I found that the window of opportunity for reversion
was highest the larger the population size,
but compensatory adaptation was most probable
at intermediate population sizes and lowest at the extremes.
%
Finally, I found that the higher the mutation rate,
the larger the window of opportunity reversion,
but it was smaller than expected because
the higher mutation rate caused more negative interactions
with compensatory mutations to occur.



Epistatic interactions among compensatory mutations that have
evolved in separate populations may form an intrinsic
postzygotic isolating barrier (i.e., hybrid inviability or sterility),
leading to biological speciation.
%
Indeed, about ten percent of genetic differences between species
comprise compensatory mutations.
%
In Chapter~\ref{chap:comp_spp} (p.~\pageref{chap:comp_spp}),
I tested experimentally whether compensatory adaptation
can lead to reproductive isolation (specifically, postzygotic isolation)
and whether it was more rapid and stronger
than in populations evolved through drift.
%
Surprisingly, the strength of this isolation was independent
of the effect size of the original deleterious mutations.
%
I also find that
both deleterious and compensatory mutations
contribute equally to reproductive isolation.
%
Our results suggest that compensatory adaptation may be
an important genetic mechanism of speciation,
and supports the view that intrinsic postzygotic isolation
can stem from intrinsic genetic mechanisms.



Reproductive isolation between populations often evolves as a byproduct of
independent adaptation to new environments, but the selective pressures of
these environments may be divergent (`ecological speciation') or uniform
(`mutation-order speciation').
%
In Chapter~\ref{chap:ecol_mo} (p.~\pageref{chap:ecol_mo}),
I directly compare the strength of reproductive isolation
(specifically, postzygotic) generated by
ecological and mutation-order processes.
%
I also tested the effect of gene flow as well as the dimensionality
(i.e., number of selective pressures) of the environments on the strength
of postzygotic isolation.
%
I found that ecological speciation generally formed stronger isolation than
mutation-order speciation, mutation-order speciation was more sensitive to gene
flow than ecological speciation, and environments with high dimensionality
formed stronger reproductive isolation than those with low dimensionality.
%
How various factors affect the strength of reproductive isolation
has been difficult to test in biological organisms, but the
use of artificial life, which provides its own genetic system that evolves,
allowed us to computationally test the effect of these factors more easily.



Under the Dob\-zhan\-sky-Mul\-ler model of speciation,
hybrid inviability or sterility results from the evolution
of genetic incompatibilities (DMIs) between species-specific alleles.
%
This model predicts that the number of pairwise DMIs between species
should increase quadratically through time,
but the few tests of this `snowball effect' have had conflicting results.
%
In Chapter~\ref{chap:snowball} (p.~\pageref{chap:snowball}),
I show that pairwise DMIs accumulated quadratically,
supporting the snowball effect.
%
The number of unfit hybrids has been proposed to accumulate faster
than quadratically because hybrids harbor more complex DMIs,
but I found that the accumulation was linear.
%
I show that more complex genetic interactions involved alleles
that rescued pairwise incompatibilities, explaining the discrepancy
between the accumulations of DMIs versus unfit hybrids.
%
Our results highlight the importance
of complex genetic interactions in speciation.

% Temporarily remove section number from bibliography title
\renewcommand\bibsection{\section*{\bibname}}

\bibliographystyle{apalike}
\bibliography{intro}

\renewcommand\bibsection{\section{\bibname}}

\addcontentsline{toc}{section}{Literature cited}

\end{doublespace}
