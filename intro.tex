\begin{doublespace}

\addcontentsline{toc}{chapter}{Introduction}
\chapter*{Introduction}

% Possible order:
% 1. sgv
%    - most adaptation happened through sgv
% 2. rate of compensation
%    - studied factors affecting compensation vs reversion
% 3. compensatory speciation
%    - can compensation lead to incompatibilities?
% 4. ecol vs MO
%    - ecol > MO; MO sensitive to gene flow
% 5. snowball
%    - quadratic increase in DMIs... maybe linear for hybrid incomp.

% SGV

Evolutionary adaptation to a new environment
depends on the availability of beneficial alleles.
%
Beneficial alleles may appear as new mutations
or may come from standing genetic variation---%
alleles already present in the population
prior to the environmental change.
%
Adaptation from standing genetic variation
in sexually-reproducing populations
is expected to be faster than from new mutations
because beneficial alleles from standing genetic variation
occur at a higher starting frequency and are immediately available.
%
The distribution of fitness effects of alleles
from standing genetic variation are expected to be different
from that of new mutations because standing genetic variation
has been `pre-tested' by selection.
%
Whether adaptation uses standing genetic variation
or new mutations as a source of beneficial alleles is unknown.
%
In Chapter~\ref{chap:sgv} (p.~\pageref{chap:sgv}),
I conducted experimental evolution of digital organisms
to determine the source of beneficial alleles during adaptation.
%
I also tested the speed of adaptation
and the fitness effect of alleles
under these two sources of genetic variation.



In Chapter~\ref{chap:sgv} (p.~\pageref{chap:sgv}),
I explored the importance of standing genetic variation
in adaptation to a new environment.
%
I determined the frequency in which adaptation
proceeded via standing genetic variation vs. new mutations
as well as the extent and speed of adaptation
from these two sources of alleles.
%
In Chapter~\ref{chap:comp_rate} (p.~\pageref{chap:comp_rate}),
I estimated the probability of compensatory adaptation
under various experimental conditions.
%
I varied three factors important to the dynamics of compensatory adaptation:
the fitness effect the initial deleterious mutation,
the population size, and the mutation rate.
%
In Chapter~\ref{chap:comp_spp} (p.~\pageref{chap:comp_spp}),
I tested experimentally whether compensatory adaptation
can lead to reproductive isolation (specifically, postzygotic isolation).
%
I also measured the speed at which isolation developed
through compensatory adaptation and the strength of genetic incompatibilities
that formed among compensatory mutations.
%
In Chapter~\ref{chap:ecol_mo} (p.~\pageref{chap:ecol_mo}),
I compared the strength of postzygotic isolation that formed
between populations that evolved in either different environments
(`ecological speciation') or parallel environments
(`mutation-order speciation').
%
For these two modes of speciation,
I also tested their sensitivity to migration between populations
and their efficacy under few or many agents of selection (i.e., dimensionality).
%
Finally, in Chapter~\ref{chap:snowball} (p.~\pageref{chap:snowball}),
I determined whether pairs of diverging populations accumulate
genetic incompatibilities between alleles at a quadratic rate through time
(the `snowball effect').
%
I compared this relationship to the accumulation of hybrid inviability
through time.

\end{doublespace}
