\begin{abstract}

Detailed experimental studies in evolutionary biology
are sometimes difficult---even with model organisms.
%
Theoretical models alleviate some of these difficulties
and often provide clean results,
but they cannot always capture the complexity
of dynamic evolutionary processes.
%
Artificial life systems are tools that fall somewhere
between model organisms and theoretical models
that have been successfully used to study evolutionary biology.
%
These systems simulate simple organisms that replicate,
acquire random mutations, and reproduce differentially;
as a consequence, they evolve naturally
(i.e., evolution itself is not simulated).
%
Here I use the software Avida to study several open questions
on the genetic mechanisms of adaptation and speciation.



In Chapter~\ref{chap:sgv} (p.~\pageref{chap:sgv}),
I investigated whether beneficial alleles during adaptation
came from new mutations or standing genetic variation---%
alleles already present in the population.
%
I found that most beneficial alleles came from standing genetic variation,
but new mutations were necessary for long-term evolution.
%
I also found that adaptation from standing genetic variation
was faster than from new mutations.
%
Finally, I found that recombination brought together
beneficial combinations of alleles from standing genetic variation.



In Chapter~\ref{chap:comp_rate} (p.~\pageref{chap:comp_rate}),
I investigated the probability of compensatory adaptation vs. reversion.
%
Compensatory adaptation is the fixation of mutations
that ameliorate the effects of deleterious mutations
while the original deleterious mutations remain fixed.
%
I found that compensatory adaptation was very common,
but the window of opportunity for reversion was increased when
the initial fitness of the population was high,
the population size was large, and the mutation rate was high.
%
The reason that the window of opportunity for reversion was constrained
was that negative epistatic interactions with compensatory mutations
prevented the revertant from being beneficial to the population.



In Chapter~\ref{chap:comp_spp} (p.~\pageref{chap:comp_spp}),
I showed experimentally that compensatory adaptation
can lead to reproductive isolation (specifically, postzygotic isolation).
%
In addition, I found that the strength of this isolation was independent
of the effect size of the original deleterious mutations.
%
Finally, I found that both deleterious and compensatory mutations
contribute equally to reproductive isolation.



Reproductive isolation between populations often evolves as a byproduct of
independent adaptation to new environments, but the selective pressures of
these environments may be divergent (`ecological speciation') or uniform
(`mutation-order speciation').
%
In Chapter~\ref{chap:ecol_mo} (p.~\pageref{chap:ecol_mo}),
I compared directly the strength of postzygotic isolation
generated by ecological and mutation-order processes
with and without migration.
%
I found that ecological speciation generally formed stronger isolation
than mutation-order speciation
and that mutation-order speciation was more sensitive to migration
than ecological speciation.



Under the Dob\-zhan\-sky-Mul\-ler model of speciation,
hybrid inviability or sterility results from the evolution
of genetic incompatibilities (DMIs) between species-specific alleles.
%
This model predicts that the number of pairwise DMIs between species
should increase quadratically through time,
but the few tests of this `snowball effect' have had conflicting results.
%
In Chapter~\ref{chap:snowball} (p.~\pageref{chap:snowball}),
I show that pairwise DMIs accumulated quadratically,
supporting the snowball effect.
%
I found that more complex genetic interactions involved alleles
that rescued pairwise incompatibilities, explaining the discrepancy
between the expected accumulations of DMIs and observation.

\end{abstract}
